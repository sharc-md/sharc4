\documentclass[11pt]{beamer}

\usepackage{libertine}
\usepackage[libertine]{newtxmath}
\usepackage{beramono}
\usepackage[utf8x]{inputenc}
\usepackage[T1]{fontenc}


\usepackage{tikz,graphicx,siunitx,pgfplots,amsmath}
\usepackage{mystuff}
\usepackage[version=3]{mhchem}
\usepackage[active,tightpage]{preview}
\PreviewEnvironment{tikzpicture}
\setlength\PreviewBorder{1pt}
\setlength{\textwidth}{246pt}
\pgfplotsset{every tick label/.append style={/pgf/number format/1000 sep={}}}
\pgfplotsset{compat=1.3}
\usetikzlibrary{plotmarks,fit,positioning}
\DeclareSIUnit\year{y}
\usetikzlibrary{calc}
\usetikzlibrary{decorations}

% \usepackage{sfmath}












\begin{document}
  \begin{tikzpicture}[font=\sffamily, >=stealth, scale=1, thick]



\tikzset{
    every node/.style={
        font=\bfseries,
        rectangle, rounded corners, draw=B, fill=B!20, anchor=center,
        minimum height=0.6cm
    },
}

  \node (sh)  at (  0,0) {sharc.x};
  \node (qc)  at (  4,0) {Interface};

  \node[anchor=west] (qc1)  at (  6,+2.1) {\textsc{Molpro}};
  \node[anchor=west] (qc2)  at (  6,+1.5) {\textsc{Molcas}};
  \node[anchor=west] (qc3)  at (  6,+0.9) {\textsc{Columbus}};
  \node[anchor=west] (qc4)  at (  6,+0.3) {Analytical};
  \node[anchor=west] (qc5)  at (  6,-0.3) {ADF};
  \node[anchor=west] (qc6)  at (  6,-0.9) {\textsc{Turbomole}};
  \node[anchor=west] (qc7)  at (  6,-1.5) {\textsc{Gaussian}};
  \node[anchor=west] (qc8)  at (  6,-2.1) {LVC};

  \node[anchor=west] (aux1) at (  5,-2.9) {\textsc{Wfoverlap}};
  \node[anchor=west] (aux2) at (  5,-3.5) {\textsc{TheoDORE}};



\tikzset{
  every node/.style={
    font=,
    rectangle, rounded corners, draw=G, fill=G!20, anchor=center,
    minimum height=0.6cm
  },
}

%   \node (i1) at (-0.5,2.0) {input};
%   \node (i2) at (-0.5,2.6) {geom};
%   \node (i3) at (-0.5,3.2) {veloc};

\tikzset{
  every node/.style={
    font=,
    rectangle, rounded corners, draw=Y, fill=Y!20, anchor=center,
    minimum height=0.6cm
  }
}

%   \node (o1) at (+0.5,-2.0) {restart.*};
%   \node (o2) at (+0.5,-2.6) {output.lis};
%   \node (o3) at (+0.5,-3.2) {output.log};
%   \node (o4) at (+0.5,-3.8) {output.dat};
%   \node (o5) at (+0.5,-4.4) {output.xyz};

\tikzset{
  every node/.style={
    font=,
    rectangle, rounded corners, draw=black, fill=black!10, anchor=center,
    minimum height=0.6cm
  }
}

  \node (q1) at (2.0,1) {QM.in};
  \node (q2) at (2.0,-1) {QM.out};

\tikzset{
  every node/.style={
    font=,
    rectangle, rounded corners, draw=G, fill=G!20, anchor=center,
    minimum height=0.6cm
  }
}

  \node[anchor=east] (t1) at (4.0,3.2) {template};
  \node[anchor=east] (t2) at (4.0,2.6) {resources};
  \node[anchor=east] (t3) at (4.0,2.0) {initial MOs};







  \draw[->] (sh) to (q1.west);
  \draw[->] (q1.east) to (qc);
  \draw[->] (qc) to (q2.east);
  \draw[->] (q2.west) to (sh);

  \draw[->] (t1.east) to[bend left] (qc.70);
  \draw[->] (t2.east) to[bend left] (qc.90);
  \draw[->] (t3.east) to[bend left] (qc.110);

  \draw[<->] (qc.21) -- (qc1.west);
  \draw[<->] (qc.15) -- (qc2.west);
  \draw[<->] (qc.9) -- (qc3.west);
  \draw[<->] (qc.3) -- (qc4.west);
  \draw[<->] (qc.357) -- (qc5.west);
  \draw[<->] (qc.351) -- (qc6.west);
  \draw[<->] (qc.345) -- (qc7.west);
  \draw[<->] (qc.339) -- (qc8.west);

  \draw[<->] (qc.280) to[bend right] (aux1.west);
  \draw[<->] (qc.260) to[bend right] (aux2.west);

  \end{tikzpicture}
\end{document}